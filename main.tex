\documentclass[11pt]{article}
\usepackage[utf8]{inputenc}	
\usepackage{amsmath,amsthm,amsfonts,amssymb,amscd}
\usepackage{multirow,booktabs}
\usepackage[table]{xcolor}
\usepackage{fullpage}
\usepackage{lastpage}
\usepackage{enumitem}
\usepackage{fancyhdr}
\usepackage{mathrsfs}
\usepackage{wrapfig}
\usepackage{setspace}
\usepackage{calc}
\usepackage{multicol}
\usepackage{cancel}
\usepackage[retainorgcmds]{IEEEtrantools}
\usepackage[margin=3cm]{geometry}
\usepackage{amsmath}
\newlength{\tabcont}
\setlength{\parindent}{0.0in}
\setlength{\parskip}{0.05in}
\usepackage{empheq}
\usepackage{framed}
\usepackage[most]{tcolorbox}
\usepackage{xcolor}
\colorlet{shadecolor}{orange!15}
\parindent 0in
\parskip 12pt
\geometry{margin=1in, headsep=0.25in}
\theoremstyle{definition}
\newtheorem{defn}{Definition}
\newtheorem{reg}{Rule}
\newtheorem{exer}{Exercise}
\newtheorem{note}{Note}
\begin{document}
\title{Code and Data for the Social Sciences: A Practitioner's Guide}



\begin{center}
{\LARGE \bf Code and Data for the Social Sciences: A Practitioner's Guide}\\
\end{center}


\section{Rules}

\begin{enumerate}
\item Automate everything that can be automated.
\item Write a single script that executes all code from beginning to end.

\item Store code and data under version control. 
\item Run the whole directory before checking it back in.

\item Separate directories by function. 
\item Separate files into inputs and outputs. 
\item Make directories portable.

\item Store cleaned data in tables with unique, non-missing keys. 
\item Keep data normalized as far into your code pipeline as you can.

\item Abstract to eliminate redundancy. 
\item Abstract to improve clarity. 
\item Otherwise, don’t abstract.

\item Don’t write documentation you will not maintain. 
\item Code should be self-documenting.

\item Manage tasks with a task management system. 
\item E-mail is not a task management system.


\end{enumerate}
\subsection{Acceleration Without Rotation}
Consider an inertial reference frame (i.e not accelerating) which will be denoted S$_0$, and a accelerating reference frame, \textit{S} that has an acceleration of \textit{A}. 


\begin{note}
\textbf{Capital Letters refer to the accelerating reference frame \textit{S} while lowercase letters refer to the inertial reference frame S$_0$}
\end{note}



\begin{shaded}
\textbf{The Tidal Force} \newline
\begin{equation}
F_{tide} = -GM_mm(\frac{\hat{d}}{d^2}-\frac{\hat{d_0}}{d_0^2})
\end{equation}
Where:
\begin{equation*}
\begin{split}
G = \text{Gravitational Constant} \\
d = \text{Object's Position Relative to Moon} \\
d_0 = \text{Earth's Center Relative to the moon}\\
M_m = \text{Mass of the moon}
\end{split}
\end{equation*}
\end{shaded}

\end{document}